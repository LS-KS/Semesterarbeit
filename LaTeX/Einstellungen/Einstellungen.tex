% !TeX spellcheck = de_DE

% Do's & Dont's in LaTeX
%\RequirePackage[l2tabu,orthodox]{nag}

% Deutsch und UTF-8 Encoding
\usepackage[english,ngerman]{babel}
\selectlanguage{ngerman}

\usepackage[utf8]{inputenc}
\usepackage[T1]{fontenc}
\usepackage{lmodern}

% Seiteneinstellungen
\usepackage{geometry}
\geometry{left=35mm,right=35mm, bindingoffset=0mm, top=30mm,bottom=30mm}

% Abstand im Text
\usepackage{xspace}
\usepackage{setspace}

% Kapitelüberschrift: große Nummer -> Titel
\renewcommand*{\chapterformat}{\mbox{\chapappifchapterprefix{\nobreakspace}\scalebox{3}{\thechapter}\enskip}}

% Linie unter \chapter
\makeatletter
\renewcommand{\chapterlinesformat}[3]{%
  \parbox[t]{\linewidth}{%
    \raggedchapter\@hangfrom{#2}{#3}\par%
    \vspace*{-.75\ht\strutbox}\rule{\linewidth}{.8pt}%
  }%
}
\makeatother

% pdf Pakete
\usepackage[pdfusetitle=true,colorlinks=true,linkcolor=blue,urlcolor=blue,citecolor=blue,bookmarks=true,bookmarksopenlevel=2]{hyperref}

% Pakete für Grafiken
\usepackage{graphicx}
\usepackage{color}
\usepackage{xcolor}
\usepackage{tikz}
\usepackage{pgfplots}
\pgfplotsset{compat=1.5}
\usepackage{caption}
\captionsetup{labelfont=bf, textfont=small} 
\usepackage{subcaption}

% Tabellentools
\usepackage{tabularx}
\usepackage{supertabular}
\usepackage{multirow}
\usepackage{multicol}

% Mathematik
\usepackage[fleqn]{amsmath}
\usepackage{amssymb,amsthm,textcomp}
\usepackage{upgreek}

% Zahlenformate
\usepackage{siunitx}
\sisetup{range-phrase=\ \textrm{bis}\ }
\sisetup{locale=DE}

% Auflistungen
\usepackage{enumerate}

% Symbolverzeichnis
\usepackage[nomentbl]{Einstellungen/nomencl-table}

% Glossar
%\usepackage[nonumberlist,toc,nopostdot,style=alttree,xindy]{glossaries}
\usepackage{glossaries}

% Index
\usepackage{makeidx}

% Listings
\usepackage{listings}

% Eigene Tabellenspalten
\usepackage{tabularx}
\usepackage{dcolumn}

%%%%%%%%%%%%%%%%%%%%%%%%%%%%%%%%%%%%%%%%%%%%%%%%%%%%%%%%%%%%%%%%%%%%%%%%%%%%%%
% HIER EIGENE PAKETE HINZUFÜGEN
%
%
%
%
%%%%%%%%%%%%%%%%%%%%%%%%%%%%%%%%%%%%%%%%%%%%%%%%%%%%%%%%%%%%%%%%%%%%%%%%%%%%%%
% Vor dem Laden von weiteren Paketen in das Latex-Sündenregister von Marc Ensenbach schauen!
%%%%%%%%%%%%%%%%%%%%%%%%%%%%%%%%%%%%%%%%%%%%%%%%%%%%%%%%%%%%%%%%%%%%%%%%%%%%%%

%%%%% Sonstiges

% Es können todo-Notes eingefügt werden
\usepackage{todonotes}

% Definition der Kopf und Fußzeilen
\usepackage{fancyhdr,lastpage}
\fancypagestyle{fancy2}{%
	\fancyhf{}
	\fancyhead[L]{}
	\fancyhead[C]{}
	\fancyhead[R]{}
	\fancyfoot[OL]{}
	\fancyfoot[EL]{\thepage}
	\fancyfoot[C]{ }
	\fancyfoot[OR]{\thepage}
	\fancyfoot[ER]{}
	\renewcommand{\headrulewidth}{1pt}
	\renewcommand{\footrulewidth}{1pt}
}
\fancypagestyle{plain}{%
	\fancyhf{}%
	\renewcommand{\headrulewidth}{0pt}%
	\renewcommand{\footrulewidth}{0pt}
	\fancyfoot[OR]{\thepage}
}
\renewcommand{\headrulewidth}{0pt}
\renewcommand{\footrulewidth}{0pt}
\usepackage{emptypage}

% \matr{A} ergibt "fettes A" wie es bei Matrizen aussehen soll
\newcommand{\matr}[1]{\mathbf{#1}}

% Definition der Uni Kassel Farbpalette
\definecolor{UKDunkelGrau}{RGB}{87,87,87}
\definecolor{UKMittelGrau}{RGB}{157,157,157}
\definecolor{UKHellGrau}{RGB}{218,218,218}
\definecolor{UKDunkelPink}{RGB}{154,12,70}
\definecolor{UKMittelPink}{RGB}{199,16,92}
\definecolor{UKHellPink}{RGB}{243,216,221}
\definecolor{UKGruen}{RGB}{21,56,36}
\definecolor{UKBlau}{RGB}{80,149,200}
\definecolor{UKGelb}{RGB}{196,210,15}
\definecolor{UKTuerkis}{RGB}{74,172,150}
\definecolor{UKGold}{RGB}{234,195,114}

%%%%%%%%%%%%%%%%%%%%%%%%%%%%%%

\renewcommand{\familydefault}{\rmdefault}
\renewcommand*\rmdefault{ppl}

\setlength{\parindent}{0pt}
\setlength{\parskip}{6pt}

\makeatletter
% !TeX spellcheck = de_DE
\renewcommand{\maketitle}{\fontfamily{cmss}\selectfont
  \begin{center}
	\vspace*{-0.5cm}
	\includegraphics[width=75mm]{Bilder/UniKS-Logo}\\

	\vspace{1.5cm}
	\centering
        \ifthenelse{\equal{\Typ}{MSc}}{
	% Masterarbeit
	\textbf{\Large Masterarbeit}\\[1ex]
        }{\ifthenelse{\equal{\Typ}{BSc}}{
	% Bachelorarbeit
	\textbf{\Large Bachelorarbeit}\\[1ex]
        }{
        %  Restliche Arbeiten:
	\textbf{\Large \Typ}\\[4ex]
        }}
	\textsf{\Huge \@title}\\[4ex]
%
	{\Large \@author}\\[4ex]
%
        \ifthenelse{\isundefined{\Titelgrafik}}{}{
  	  \vfill
          \includegraphics[height=8cm]{\Titelgrafik}
  	  \vfill
        }
	\vfill
        \ifthenelse{\equal{\Typ}{Technischer Bericht}}{
        Bericht-Nr. \MRTnr\\  
        }{ 
        \begin{center}
          %\fbox{\parbox{14cm}{
          \renewcommand{\arraystretch}{1.2}
		\begin{tabularx}{\columnwidth}{p{0.5\textwidth}>{\raggedleft\arraybackslash}p{0.5\textwidth}}
			Matrikelnummer: & \MatNr\\
                        \ifthenelse{\equal{\Typ}{MSc}\or\equal{\Typ}{BSc}}{
			Erstgutachter: & \Erstgutachter\\
			Zweitgutachter: & \Zweitgutachter\\}{
			Gutachter: & \Erstgutachter\\}
                        \ifthenelse{\isundefined{\Betreuer}}{}{
			Betreuer: & \Betreuer\\}
			Tag der Abgabe: & \@date\\
                        \ifthenelse{\isundefined{\MRTnr}}{}{MRT-Nr.:& \MRTnr\\}
                      \end{tabularx}%
            %          }}
                    \end{center}
        }
	\vfill
	\includegraphics[height=12mm]{Bilder/MRT-Logo} \hfill \includegraphics[height=12mm]{Bilder/ISAC-Logo}
\end{center}
}


%%% Local Variables:
%%% mode: latex
%%% TeX-master: "../MRT-Bericht-2020"
%%% End:

\makeatother

\newcounter{SeitenzahlSpeicher}

%%% Local Variables:
%%% mode: latex
%%% TeX-master: "../MRT-Bericht-2020"
%%% End:
