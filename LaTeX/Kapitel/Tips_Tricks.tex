\chapter{Tips \& Tricks}

Do's and Dont's in \LaTeX:
\url{https://mast.queensu.ca/~andrew/LaTeX/latex-dos-and-donts.pdf}

\section{Makros}
\label{sec:makros}

\TeX-\Index{Makros} \textsf{def} oder \LaTeX-Makos \textsf{newcommand}
definieren für einheitliche Schreibweisen im Dokument:
%
\begin{verbatim}
\newcommand{\Matlab}{\textsc{Matlab}\xspace}
\def\squared#1{\ensuremath{#1^2}}
\newcommand{\Makro}[3]{Argument #1 mit Argument#2 und Argument #3}

Superskript mitten im Text ohne \$\$ -> \squared{\pi}

Mit \Matlab kann man es berechnen!

\end{verbatim}
\newcommand{\Matlab}{\textsc{Matlab}\xspace}
\def\Makro#1#2#3{Argument #1 mit Argument#2 und Argument #3}
\def\squared#1{\ensuremath{#1^2}}

Superskript mitten im Text ohne \$\$ -> \squared{\pi}

Mit \Matlab kann man es berechnen!


\section{Tabellen}

Eigene Formate für Spalten in \Index{Tabellen} mit dem Paket
\textsf{tabularx} definieren:
%
\begin{verbatim}
\newcolumntype{R}{>{\raggedleft\arraybackslash}X}
\end{verbatim}

\begin{table}[h!!]
 \caption{Neuer Spaltentyp}
 \label{tab:tabularx}
  \centering
  \newcolumntype{R}{>{\raggedleft\arraybackslash}X}%
 \begin{tabularx}{\textwidth}{ |l|R|l|R| }
  \hline
  label 1 & label 2 & label 3 & label 4 \\
  \hline 
  item 1  & item 2  & item 3  & item 4  \\
  \hline
\end{tabularx}
\end{table}


Zahlen ausrichten in Spalten mit Paket \textsf{dcolumn} (und
gleichzeitig ersetzen des Dezimaltrenners ``.'' durch ein Komma ``,'')

\begin{verbatim}
\newcolumntype{d}[1]{D{.}{,}{#1} }
\end{verbatim}

\begin{table}[h!!]
\newcolumntype{d}[1]{D{.}{,}{#1} }
  \centering
  \begin{tabular}{ld{2}}
    Symbol & Wert \\\hline
    $\pi $ & 3.1415 \\
     -e    & -2.71 \\
  \end{tabular}
  \caption{Neue Spaltentypen}
  \label{tab:newcolumntype}
\end{table}

% ToDo: longtable, sidewayfigure

Hilfe: \url{https://en.wikibooks.org/wiki/LaTeX/Tables}

\section{Symbolverzeichnis}

Ein \Index{Symbolverzeichnis} kann automatisch mit dem Paket \textsf{nomencl}
angelegt werden: 

Mit dem Befehl \verb#\Symbol{symbol}{bedeutung}# wird das Symbol im
Text angegeben und gleichzeitig in das Symbolverzeichnis aufgenommen:
%
\begin{verbatim}
Die Zahl \Symbol{$\pi$}{Kreiszahl} ist transzendent.
\end{verbatim}

Die Zahl \Symbol{$\pi$}{Kreiszahl} ist transzendent.

Es können auch Einheit und Kommentar angegeben werden: 
\begin{verbatim}
Symbol mit Text: Die
\SymbolT{$g$}{Erdbeschleunigung}
wirkt immer.

Symbol mit Text und Einheit: Die 
\SymbolU{$g$}{Erdbeschleunigung}{m/s^2} 
wirkt immer.

Mit Text, Einheit und Bemerkung: Die 
\SymbolB{$c$}{Lichgeschwindigkeit}{m/s^2}{im Vakkum} 
ist maximal.
\end{verbatim}

Symbol mit Text: Die
\SymbolT{$g$}{Erdbeschleunigung}
wirkt immer.

Symbol mit Text und Einheit: Die 
\SymbolE{$g$}{Erdbeschleunigung}{m/s^2} 
wirkt immer.

Mit Text, Einheit und Bemerkung: Die 
\SymbolB{$c$}{Lichgeschwindigkeit}{m/s^2}{im Vakkum} ist maximal.


Das Symbolverzeichnis muss mit dem Befehl \textsf{makeindex} manuell
erstellt werden:
%
\begin{verbatim}
latex MRT-Bericht.tex
makeindex MRT-Bericht.nlo -s nomencl.ist -o MRT-Bericht.nls
latex MRT-Bericht.tex
\end{verbatim}

Hilfe: \url{https://www.ctan.org/pkg/nomencl?lang=de}

\section{Abkürzungsverzeichnis}

Ein \Index{Abkürzungsverzeichnis} wird in \LaTeX als \Index{Glossar}
mit dem Befehl

\verb#\Glossar{abkuerzung}{kurz}{lang}# erstellt:
%
\begin{verbatim}
Hier ein Eintrag für das AbkVz: 
\Glossar{svm}{SVM}{Support Vector Machine}
\end{verbatim}

Hier ein Eintrag für das AbkVz:
\Glossar{svm}{SVM}{Support Vector Machine}

Hilfe: \url{https://en.wikibooks.org/wiki/LaTeX/Glossary}

\section{Index}

\LaTeX erstellt automatisch einen \Index{Index} mit dem Befehl
\verb#\Index{eintrag}# angelegt:

\begin{verbatim}
\LaTeX erstellt automatisch einen Index-Eintrag mit dem Befehl
``\Index{Index}´´.
\end{verbatim}

\LaTeX{} erstellt automatisch einen Index-Eintrag mit dem Befehl
``\Index{Index}''.


Hilfe: \url{https://en.wikibooks.org/wiki/LaTeX/Indexing}


\section{Nützliche Umgebungen}

\subsection{Programm-Quellcode}

Listings können mit dem Paket \Index{\textsf{listings}} in ein Dokument
aufgenommen werden.

\begin{lstlisting}[numbers=right]
a = 1
b = 2
c = sqrt( a^2 + b^2 )
\end{lstlisting}


Man kann auch Befehle wie \lstinline{var i = 0;} direkt im Text angeben!


Dateien automatisch in das Dokument einfügen:

\begin{verbatim}
\lstinputlisting[language=Python]{Demo1.py}
\end{verbatim}


Demo 1 als Python-Quelltext
:
\lstinputlisting[language=Python]{Listings/Demo1.py}

Demo 2 mit Zeilennummern:
\lstinputlisting[numbers=left]{Listings/Demo2.py}


Hilfe: \url{https://en.wikibooks.org/wiki/LaTeX/Source_Code_Listings}

\section{Mathematische Umgebungen}

Schreibweise von Vektoren/Matrizen vereinfachen:

\def\vk#1{\mathbf{#1}}
\def\tr#1{\ensuremath{#1^{\mathrm{T}}}}
\def\zvektor#1{\left(\begin{array}{c}#1\end{array}\right)}
\def\svektor#1{\left(\begin{array}{c@{,\:}c@{,\:}c}#1\end{array}\right)}
\def\matrix#1{\left(#1\right)}

\Index{Vektor}
\begin{align}
  \tr{\vk{v_s}} = \svektor{11&12&13}
\\
\vk{v_z} = \zvektor{11\\21\\31}
\end{align}

g ist \Einheit{9.8055}{m/s^2}

Die Umgebungen des Paketes \textsf{amsmath} verwenden weniger
vertikalen Platz und können ausgerichtet werden

\begin{verbatim}
\begin{align}\label{eq:1}
  a &= 1\\
  xyz &= \i
\end{align}

\begin{alignat*}{2}
    \dot{x} &= A \cdot x &+& B \cdot u\\
          y &= C \cdot x &+& D \cdot u
\end{alignat*}

\end{verbatim}

\begin{align}\label{eq:1}
    \dot{x} &= A(t) \cdot x + B \cdot u\\
          y &= C \cdot x + D \cdot u
\end{align}
%
\def\vx{\vk{x}}
\begin{alignat*}{2} % n=2 -> n+1=3 Tabulatoren "&" einfügen
    \dot{\vx} &= \vk{A}(t) \cdot \vx  &+& B \cdot u\\
           y &=  \vk{C}    \cdot \vx  &+& D \cdot u
\end{alignat*}

%%% Local Variables:
%%% mode: latex
%%% TeX-master: "../MRT-Bericht-2020"
%%% End:
