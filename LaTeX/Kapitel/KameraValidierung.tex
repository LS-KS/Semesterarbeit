\chapter{Ideensammlung zu kameragestützten Validierungsprozessen in der Lagerverwaltung}\label{Kap5}

Die neue Softwareanwendung soll die Möglichkeit bieten, das Inventar der Lagerzelle automatisiert zu überprüfen. 
Zu diesem Zweck sollen kameragestützzte Validierungsprozesse implementiert und untersucht werden. 
Dieses Kapitel soll dafür ein Konzept entwickeln und offensichtliche Anforderungen an die Kamera ableiten.

\subsection {Konzepte}

Die Lagerzelle beinhaltet ein Regal und einen Kommissionier-Tisch, der Paletten und Becher mit ggf. Produkten enthält.
Produkte werden dabei durch Wasser, die mit farbigen Kugeln vermengt sind, simuliert.
Ein Produkt wird von einem mobilen Roboter in einem Becher angeliefert. Anschließend wird es von dem Roboterarm in das Regal geräumt und verbleibt dort bis zu seiner Auslieferung.
Die Softwareanwendung merkt sich den Inventarbestand und seine Position im Regal.
Eine Manipulation des Inventars ist durch eine Person jederzeit möglich.
Ebenso eine Fehleingabe. 

Für eine kameragestützte Inventur gibt es somit zwei Usecases: 
Zum einen könnte eine Inventur auf Anforderung eines Benutzers durchgeführt werden um einen Soll-Ist Abgleich des realen Inventars mit 
dem gespeicherten Stand der Softwareanwendung vorzunehmen. 
Zum Beispiel, weil nicht mehr sichergestellt werden kann, dass die Bestandsdaten korrekt sind.
Zum anderen können die Produkte und Becher bei Anlieferung und Auslieferung automatisch verifiziert und Diskrepanzen dem Benutzer angezeigt werden.
Wenn die kameragestützte Inventur mittels einer Kamera auf dem Greifer durchgeführt wird, kann die Lagerzelle in dieser Zeit keine Transportaufträge bearbeiten.
Um einen Lagerort zu verifizieren, muss der Greifer vor die Palette fahren, kontrollieren, dass die Palette an Ort und Stelle ist, und den vorderen Becher prüfen. 
Anschließend muss der Greifer die Palette aus dem Regal holen und auf den Kommissionier-Tisch stellen.
Erst jetzt kann der Greifer den hinteren Becher prüfen und die Palette wieder in das Regal stellen.

Wenn der Roboterarm aus einer schnellen Bewegung heraus für eine Aufnahme stoppt, ist zu erwarten, dass aufgrund von Schwingungen nicht direkt mit der Bildaufnahme begonnen werden.
Eine Bildaufnahme an sich kann ebenfalls bis zu 5 Sekunden dauern. 
Wenn man überschlägt, dass für die Anfahrt, Auslagern und Einlagern je 5 Sekunden benötigt werden (was eine eher optimistische Schätzung ist), dauert eine Inventur pro Lagerort 25 Sekunden.
Bei 18 Paletten, ist die Lagerzelle somit für 7,5 Minuten blockiert.
Soll dazu noch das Produkt bewertet werden, muss der Greifer, nach dem die Palette auf dem Kommissionier-Tisch abgestellt wurde, von oben in den Bescher schauen und die Farben der drei 
Kugeln erkennen.
Eine andere Möglichkeit ist, eine Übersichtskamera zu verwenden. 
Diese ist an einem Ort in der Lagerzelle montiert, von dem aus alle Lagerorte, möglichst gut, überblickt werden können.
So kann während die Lagerzelle in Betrieb ist, das Inventar überprüft werden.
Bedingung dafür ist, dass die optischen Marker gut sichtbar und hoch genug aufgelöst sind.
Selbst wenn nicht alle optischen Marker von der Übersichtskamera erkannt werden können, so können dennoch von den erkannten Markern Rückschlüsse auf die Gültigkeit des Inventars gezogen werden. 
Oder aber die Daten, die mit der Übersichtskamera gewonnen werden, reduzieren die Zeit, die für die Inventur mit der Greifer-Kamera benötigt wird.


\subsection {Abgeleitete Anforderungen an die Kamera}

Für die Kameras, die für die kameragestützte Inventur verwendet werden, lassen sich aus dem Konzept Anforderungen ableiten. 
Der Greifer ist hoch beweglich und bietet aufgrund seiner Kinematik wenig Platz für eine Kamera. 
Die Kamera kann daher nur direkt auf dem Greifer befestigt werden. 
Ist ein Becher im Greifer ergibt sich daraus eine sehr kurze Distanz zwischen dem Kamerasensor und dem optischen Marker (etwa 30mm, je nach Kamera).
Die in \cite[Sebastian Hüblers Arbeit]{Hübler2019} verwendete $\mu$Eye Kamera konnte mit dem Abstand gut umgehen, wenngleich die optische Marker nicht mehr scharf sichtbar waren.
Leider ist diese Kamera defekt und es bedarf daher einer Neuanschaffung. 

An die Auflösung der Kamera auf dem Greifer gibt es keine besonderen Anforderungen, da die Marker erwartungsgemäß beinahe den gesamten Bildbereich einnehmen werden. 
Soll das Produkt erkannt werden, muss die Kamera einen RGB-Sensor haben.
Am Arm des Greifers ist bereits eine USB 2.0 Leitung verlegt, die benutzt werden soll. 

Eine Übersichtskamera wird für die größtmögliche Übersicht in eine der oberen Raumecken der Lagerzelle montiert.
Von dort aus, ist die Mitte des Regals etwa 3m entfernt. 
Bei einem arUco Marker mit einer 6x6 Binärmatrix und einer Kantenlänge von ca. 30mm, ist ein Bit etwa 3x3mm groß.
Da das Regal nicht den gesamten Bildbereich einnimmt, ist je nach Apertur der Kamera eine Auflösung von mehr als 20MP nötig. 
Zu dieser Auffassung komme ich nach Rücksprache mit dem Kamerahersteller IDS. 

Eine Kamera in der Raumecke muss sowohl eine Datenverbindung als auch eine Stromversorgung zur Verfügung gestellt bekommen.
Hochauflösende Kameras sind in der Regel mit GigE verfügbar und können über PoE mit Spannung versorgt werden. 
Eine Kamera, die beides erfüllt, spart die Verlegung einer weiteren Leitung. 


\subsection{Anforderungen an die Softwareanwendung}

Die Softwareschnittstelle der Kameras muss entweder OpenCV kompatibel sein oder eine Python-API bereitstellen.
Da einzelne Bilder verarbeitet werden sollen, ist eine hohe Bildrate nicht notwendig.
Zunächst wird mit der Übersichtkamera eine Bildaufnahme gemacht und daraus mit den Methoden der Bildverarbeitung alle optischen Marker erkannt - wo möglich.
Ein Vergleich mit dem Datenmodell ergibt die Positionen, bei denen eine Erkennung nicht funktionierte.
Für diese Positionen muss anschließend der Greifer die Inventurschritte übernehmen.
Erreicht der Roboter einen Ort von dem aus ein Bild aufgenommen werden soll, so muss er an die Softwareanwendung ein Signal senden, welches
der \verb|ABBController| verarbeitet und eine Bildaufnahme auslöst.
Während der Implementierung ist zu untersuchen, ob die Signale des Roboters Informationen über die Position enthalten müssen, oder ob die Position des Greifers aus dem Algorithmus abgeleitet werden kann.
Anschließend muss der Erkennungsalgorithmus ausgeführt werden.
Diese Schritte werden wiederholt, bis alle Lagerorte überprüft wurden.  
Sind die Inventurdaten vollständig, müssen die Daten mit den gespeicherten Daten verglichen werden.
Aus den Bildern und den daraus erkannten Daten muss eine Übersichtsansicht erstellt und angezeigt werden. 
Hier soll der Soll/Ist-Vergleich veranschaulicht werden. 
Bei Diskrepanzen soll ein Benutzer die Möglichkeit haben die Daten vereinzelt zu korrigieren, die Übernahme zu bestätigen oder zu verwerfen.


