\chapter{Kurzbeschreibung der Lagerzelle}\label{Lagerzelle}

    Die Lagerzelle ist ein abgesperrter Bereich innerhalb der $\mu$Plant. 
    Sie hat eine überwachte Zugangstür für Personen und eine Andockstation für einen mobilen Roboter, die für Personen nicht passierbar ist. 
    Die Andockstation ist mit einem RFID-Lesegerät der Fa. FEIG GmbH und zwei induktiven Näherungssensoren ausgestattet, sowie einer 
    Ladestation. 
    
    Direkt dahinter steht ein ABB Industrieroboter IRB 140 auf seinem Fundament. 

    Aus Sicht des Roboters um $90^\circ$ dazu befindet sich ein Kommissionier-Tisch mit Platz für bis zu zwei Paletten.
    Die Paletten-Plätze sind mit \glqq K1\grqq{} und \glqq K2\grqq{} gekennzeichnet.
    Für die Becher sind die Plätze \glqq a\grqq{} und \glqq b\grqq{} auf der vom Roboterarm abgewandten Seitde des Tisches
    angebracht.

    Wiederum $90^\circ$ versetzt zum Kommissioniertisch, also gegenüber der Andockstation befindet sich ein Lagerregal.
    Es hat drei Böden mit jeweils sechs Lagerplätzen für Paletten.
    Die Lagerplätze sind mit \glqq L1\grqq{}(oben links) bis \glqq L18\grqq{}(unten rechts) gekennzeichnet.
    An der dem Benutzer zugewandten Seite sind die Becherplätze \glqq a\grqq{}(vorne) und \glqq b\grqq{}(hinten) angebracht.
    
    Der Roboterarm hat einen pneumatischen Greifer, mit dem er eine Palette oder einen Becher aufnehmen kann. 