\begin{table}[h]
\centering
\caption{Benutzerinageben, mögliche Fehler und ihre Erkennung}
\begin{tabularx}{\textwidth}{|l|p{2cm}|p{2cm}|X|}
\hline
Art & Typ & Wertebereich & Fehlererkennung \\
\hline
IP Adresse & Formatierter String aus Integers & 0 \ldots 255 bzw. 0 \ldots 9 & Gültigkeitsprüfung bei Dateneingabe, Validierung bei Verbindungsaufbau\\
\hline
IP Port & Integer & 0\ldots65536 & Wertebereich bei Dateneingabe, Validierung bei Verbindungsaufbau\\
\hline
Produkt ID & Integer & 0\ldots99 & Eingabe anhand Dropdown Menü mit Anzeige des Produktnamens beschränkt Wertebereich auf zulässige Werte. Validierung nur im Softwarebetrieb durch Soll-Ist Abgleich.\\
\hline
Becher ID & Integer & 0\ldots99 & Eingabe kann nicht auf gültigen Wertebereich beschränkt werden. Soll-Ist-Vergleich mittels RFID ist möglich.\\
\hline
Palette & Bool & True / False & Abgleich mit Anwesenheit von Bechern, arUco Marker Erkennung mittels Kamera\\
\hline
\end{tabularx}\label{tab:Benutzereingaben}
\end{table}

