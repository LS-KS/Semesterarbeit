% !TeX encoding = UTF-8
% !TeX TS-program = pdflatex
% !TeX spellcheck = de_DE

\chapter{Motivation und Zielsetzung}

    Das Institut für Mess- und Regelungstechnik an der Universität Kassel hat in den letzten Jahren eine Modellfabrik $\mu$Plant gebaut.
    Aus über 70 Einzelarbeiten ist ein modernes Industrie-4.0 Konzept geschaffen worden.
    Teil der $\mu$Plant ist ein vollautomatisiertes Lager.
    Das Lager besteht aus einem abgetrennten Raum, dessen Zugang über eine Tür mit einem Türschalter überwacht ist.
    In diesen Bereich können autonome mobile Roboter (Turtlebots) einfahren.
    In dem abgetrennten Bereich steht ein Industrieroboter Typ ABB IRB 140 und ein Lagerregal mit ausgewiesenen 18 Lagerplätzen.
    Außerdem befindet sich neben einer Andockstation für den Turtlebot ein Kommissioniertisch. \\

    Ein pneumatischer Greifer kann Paletten, die je mit bis zu zwei Bechern bestückt werden können,
    zwischen dem mobilen Roboter und dem Lagerregal transportieren.
    Von einem PC-Arbeitsplatz aus können mittels Software die Lagerprozesse überwacht werden.
    Im Fehlerfall kann eingeschritten werden oder es können manuell Prozesse ausgelöst werden.\\

    Die Software ist derzeit in 3 Programme aufgeteilt: Einerseits gibt es die Lagerverwaltung 3.0 - die Hauptsoftware.
    Sie bildet die automatisierten Prozesse ab und verfügt über ein GUI welches u.A.\ den Bestand visualisiert.
    Daneben gibt es den Warehouse Controller, der dazu verwendet wird, manuelle Prozesse auszulösen.
    Zudem gibt es ein Programm \glqq RFID-Server\grqq mit dem über RFID Leser der Fa. Feig Tags der Transportbehälter
    ausgelesen werden können.
    \\
    Mit dem Wechsel des Betriebssystems von Windows 7 auf Windows 10 ist die Kompatibilität der C\# Implementierung
    nicht mehr gegeben.
    Außerdem laufen Teilfunktionen des Programms nicht fehlerfrei oder tolerieren kaum Fehlbedienungen.
    Die Dreiteilung der Software ist im Allgemeinen auch nicht mehr erwünscht. \\

    Diese Seminararbeit beschäftigt sich mit der Analyse der bestehenden Software:
    Es wird ermittelt, aus welchen Programmteilen und Funktionen die Software besteht.
    Aus den Erkenntnissen wird ein Konzept entwickelt, welches die Funktionen der Drei Software Teile zusammenführt.
    Dies soll die Grundlage für eine Migration der Software nach Python schaffen.

    Erkenntnisse aus der studentischen Arbeit von Sebastian Hübler aus dem Jahr 2019 \cite{Hübler2019} sollen überprüft
    und vertieft werden um Anforderungen an Kameras und arUco Marker zu ermitteln, die später eine automatisierte
    Inventur ermöglichen sollen.

