% !TeX spellcheck = de_DE

\chapter{Grundlagen}

Nachfolgend ein paar Beispiele für Formeln, Abbildungen, Tabellen und Quellenangaben.

%%%%%%%%%%%%%%%%%%%%%%%%%%%%%%%%%%%%%%%%%%%%%%%%%%
% FORMEL
%%%%%%%%%%%%%%%%%%%%%%%%%%%%%%%%%%%%%%%%%%%%%%%%%%
Eine Formel:

\begin{align}
	\dot{m}_\mathrm{CH4} \left( p_\mathrm{U}, T_\mathrm{U}, C_\mathrm{ppmm}, \overline{v} \right) = \dfrac{p_\mathrm{U} \cdot M_\mathrm{CH4}}{R \cdot T_\mathrm{U}} \dfrac{C^{*}}{\num{10}^{\num{6}}} \mathrm{D} \cdot d \cdot \tan \left( \dfrac{\alpha}{2} \right)
	\label{eq:DierksscheFormel}
\end{align}

mit dem Dierks'schen Faktor $\mathrm{D} = \num{2}$.

%%%%%%%%%%%%%%%%%%%%%%%%%%%%%%%%%%%%%%%%%%%%%%%%%%
% ABBILDUNG
% Schaubilder werden am besten mit "tikz" erstellt
% Bei Plots (z.B.) am besten aus Matlab nur die Daten übernehmen und "pgfplot" erstellen
%%%%%%%%%%%%%%%%%%%%%%%%%%%%%%%%%%%%%%%%%%%%%%%%%%

Referenzen auf eine Pixelgrafik (siehe Abbildung \ref{fig:TestPixel}),
eine zweiteilige Abbildung mit dem Paket \textit{subcaption} (siehe
Abbildung \ref{fig:TestDoppel} -- man kann auch
\ref{fig:TestDoppelLinks} und \ref{fig:TestDoppelRechts} einzeln
referenzieren), ein Schaubild mit dem Paket \textit{tikz} (siehe
Abbildung \ref{fig:TestTikz}) und ein Plot mit dem Paket
\textit{pgfplots} (siehe Abbildung \ref{fig:TestPgf}).

\begin{figure}[!htbp]
	\centering
	\includegraphics[width=0.45\textwidth]{Bilder/Testbild}
	\caption{Dies ist eine Pixelgrafik.}
	\label{fig:TestPixel}
\end{figure}

\begin{figure}[!htbp]
	\centering
	\begin{subfigure}[c]{.5\linewidth}
		\centering
		\includegraphics[width=0.9\textwidth]{Bilder/Testbild}
		\caption{Linkes Bild}
		\label{fig:TestDoppelLinks}
	\end{subfigure}%
	\begin{subfigure}[c]{.5\linewidth}
		\centering
		\includegraphics[width=0.9\textwidth]{Bilder/Testbild}
		\caption{Rechtes Bild}
		\label{fig:TestDoppelRechts}
	\end{subfigure}
	\caption{Eine Abbildung mit zwei Teilbildern.}
	\label{fig:TestDoppel}
\end{figure}

\begin{figure}[!htbp]
	\centering
	\begin{tikzpicture}
		\node[rectangle, draw = black, text centered, text width = 3.0cm, minimum height=1cm] (NodeA) {Teil 1};
		\node[rectangle, draw = black, text centered, text width = 3.0cm, minimum height=1cm, below = of NodeA] (NodeB) {Teil 2};
		\draw[-latex, thick] (NodeA) -- (NodeB) node[midway, right] {Verbindung};		
	\end{tikzpicture}
	\caption{Dies ist ein mit \textit{tikz} erstelltes Schaubild.}
	\label{fig:TestTikz}
\end{figure}

\begin{figure}[!htbp]
	\centering
	\begin{tikzpicture}
		\begin{axis}[%
			xlabel={Geschwindigkeit $v$ in \si{\kilo\meter\per\hour}},
			ylabel={Temperatur $T$ in \si{\degreeCelsius}},
			height = 0.5\textwidth,
			width = 0.8\textwidth,
			xmin=0,xmax=120,
			ymin=0,ymax=100,
			xtick={0,10,...,120},
			legend style={legend pos = south east}]%
			\addplot [color=UKDunkelPink,mark=o]%
				coordinates{
					(0,0)
					(120,100)
				};%
			\addplot [color=UKBlau,mark=x]%
				coordinates{
					(0,80)
					(120,40)
				};%
		\legend{Gerade $\num{1}$, Gerade $\num{2}$}
		\end{axis}%
	\end{tikzpicture}
	\caption{Dies ist ein mit \textit{pgfplots} (und \textit{tikz}) erstelltes Diagramm.}
\label{fig:TestPgf}
\end{figure}

%%%%%%%%%%%%%%%%%%%%%%%%%%%%%%%%%%%%%%%%%%%%%%%%%%
% TABELLE
%%%%%%%%%%%%%%%%%%%%%%%%%%%%%%%%%%%%%%%%%%%%%%%%%%

Eine Beispieltabelle ist in Tabelle \ref{tab:Test} zu finden.

\begin{table}[!htbp]
	\caption{Dies ist eine Tabelle mit \textit{multicolumn} und \textit{multirow} Beispielen.}
	\label{tab:Test}
	\centering
	\begin{tabular}{|c|c|c|}
		\hline
		\textbf{Eigenschaften} & \textbf{Gegenstand 1} & \textbf{Gegenstand 2}\\
		\hline
		$E_1$ & \multicolumn{2}{c|}{zufällig identisch}\\
		\hline
		\multirow{2}{*}{$E_2$} & \multirow{2}{*}{kurz} & Zufällig sehr\\
		 & & lang\\
		\hline
	\end{tabular}
\end{table}

%%%%%%%%%%%%%%%%%%%%%%%%%%%%%%%%%%%%%%%%%%%%%%%%%%
% AUFZÄHLUNG UND ZITATION
%%%%%%%%%%%%%%%%%%%%%%%%%%%%%%%%%%%%%%%%%%%%%%%%%%

Beispielaufzählung und Zitate:

\begin{itemize}
	\item Buch \cite{2016-Book-Kroll-CIA}
	\item Wissenschaftlicher Artikel \cite{OrdonezMueller2017}
	\item Konferenzbeitrag \cite{Schramm2019}
	\item Dissertation \cite{OrdonezMueller2018}
	\item Patent \cite{DPatent2017}
	\item Onlinequelle \cite{MRTToolbox2020}
\end{itemize}

%%% Local Variables:
%%% mode: latex
%%% TeX-master: "../MRT-Bericht-2020"
%%% End:
