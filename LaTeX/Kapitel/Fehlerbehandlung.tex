\chapter{Analyse zur Fehlerbehandlung}\label{Fehlerbehandlung}

Fehler können systematischer oder sporadischer Natur sein.
Die effektive Fehlersuche und Fehlerbehebung ist ein wichtiger Bestandteil der Softwareentwicklung, der den Entwickler 
ständig begleitet. 

\subsection{Fehleridentifikation}

Syntaxfehler und Fehler in der Benutzung der Programmiersprache Python lassen sich in der Regel gut finden und beheben.
Der größte Teil wird bereits bei der Erstellung des Quellcodes erkannt und farblich markiert. 
Andere Fehler führen bei jedem Ausführen des Codes zu einem Ausnahmefehler, den der Interpreter erkennt und (meistens) eine brauchbare
Fehlermeldung in der Konsole ausgibt. 

Weitere Fehler sind oft in der Logik des Programms zu finden. Diese zu identifizieren ist in der Regel schwieriger.
Es ist jedoch möglich, nach dem Eingabe-Verarbeitung-Ausgabe Prinzip vorzugehen. 
Dabei bewertet man engmaschig die Eingabe und die Ausgabe und vergleicht Soll mit Ist. 

Eingaben können Werte innerhalb eines Bereichs annehmen.
Hier muss der Entwickler alle erwartbaren Sonderfälle abdecken.
Für systematische Fehler können in der Softwareentwicklung Tests definiert werden, die das Programmverhalten 
für ein Spektrum an Eingaben überprüfen.
Die Ausgaben werden mit den erwarteten Ausgaben durch Assertions verglichen.
In Python ist dies beispielsweise mit dem Paket \verb|pytest| ( siehe \cite{pytestHP}) oder \verb|unittest| ( siehe \cite{unittestHP}) möglich.
Anhand einer kurzen Recherche beider Internetauftritte werde ich pytest verwenden.
Einerseits liefert pytest bei fehlgeschlagenen Tests eine ausführlichere Analysemeldung, andererseits wird pytest von Qt empfohlen.
Mit pytest lassen sich aber nur Python Module testen.

Für einen GUI-Test muss das \verb|QtTest| Framework verwendet werden.
Leider lässt sich zu diesem Paket keine umfangreiche Dokumentation. 
Ich habe versucht dieses Framework zusammen mit dem Qt Creator zu benutzen. 
Entweder erhielt ich Fehlermeldungen, die ich nicht mit meinem Code in Verbindung bringen konnte 
oder das Ausführen des Programms ignorierte die geschriebenen Tests.
Dies bewegt mich zu der Empfehlung von diesem Framework ohne weitere Updates und Dokumentation seitens Qt abzusehen.

Sporadische Fehler können nicht unbedingt vorhergesehen werden. 
Sie rühren meistens aus einer unerwarteten Kombination von Eingaben die selten auftritt oder durch einen Fehler in einem eingebundenen Dienst. 
Eine gut dokumentierte API weist auf Fehlermöglichkeiten hin und gibt Hinweise zur Fehlerbehandlung.
Die erwarteten Quellen sind Eingabefehler und Kommunikationsfehler.


\subsubsection{Erkennung fehlerhafter Benutzereingaben}

Werden im Programm falsche Daten eingegeben eingelesen, soll dies, soweit wie möglich, entdeckt und abgefangen werden.
Die Tabelle \ref{tab:Benutzereingaben} listet alle erwarteten Benutzereingaben und mögliche Methoden die Benutzereingaben
zu validieren.

\begin{table}[h]
\centering
\caption{Benutzerinageben, mögliche Fehler und ihre Erkennung}
\begin{tabularx}{\textwidth}{|l|p{2cm}|p{2cm}|X|}
\hline
Art & Typ & Wertebereich & Fehlererkennung \\
\hline
IP Adresse & Formatierter String aus Integers & 0 \ldots 255 bzw. 0 \ldots 9 & Gültigkeitsprüfung bei Dateneingabe, Validierung bei Verbindungsaufbau\\
\hline
IP Port & Integer & 0\ldots65536 & Wertebereich bei Dateneingabe, Validierung bei Verbindungsaufbau\\
\hline
Produkt ID & Integer & 0\ldots99 & Eingabe anhand Dropdown Menü mit Anzeige des Produktnamens beschränkt Wertebereich auf zulässige Werte. Validierung nur im Softwarebetrieb durch Soll-Ist Abgleich.\\
\hline
Becher ID & Integer & 0\ldots99 & Eingabe kann nicht auf gültigen Wertebereich beschränkt werden. Soll-Ist-Vergleich mittels RFID ist möglich.\\
\hline
Palette & Bool & True / False & Abgleich mit Anwesenheit von Bechern, arUco Marker Erkennung mittels Kamera\\
\hline
\end{tabularx}\label{tab:Benutzereingaben}
\end{table}



Wenn manuell Transportaufträge eingegeben werden, kann es zu Konflikten kommen.
Z.B. könnte im Abholort kein Becher oder keine Palette sein.
Umgekehrt könnte am Abstellort eine Palette oder Becher stehen.

Für diese Problematiken können :
\begin{itemize}
    \item Inventardaten zur Überprüfung herangezogen werden
    \item Kameragestützte Validierungsprozesse in Kapitel \ref{Kap5} Entworfen werden.
\end{itemize}

\subsection{Fehlerbehandlung}

Fehler müssen behoben werden, wenn erwartet wird, dass sie das Programm auf negative Weise beeinflussen.
Für das Datenmodell gilt das Prinzip, dass jede Änderung getestet werden muss. 
Der Test muss dafür für alle erwartbaren Gegebenheiten erweitert werden. 

Dies gilt auch für den invController, da dieser das Datenmodell manipuliert.

Für Controller und Services reicht es aus, die Funktionen durch Ausprobieren zu testen. 
Der Funktionsumfang eines Controllers oder Services ist meistens geringer als der eines Datenmodells.
Systematische Fehler können dadurch gut abgefangen und korrigiert werden. 
Außerdem wird so klar, welche Fälle in Exceptions abgefangen und behandelt werden müssen. 

Sporadische Fehler treten in Controllern und Services eher selten auf und können im Fall ihres Auftretens analysiert und behoben werden. 

