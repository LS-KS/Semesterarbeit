\chapter{Funktions- und Anforderungsanalyse}\label{Funktionsanalyse}

In diesem Kapitel werden auf Grundlage der vorangegangenen Kapitel die notwendigen Funktionen und ihre Anforderungen ermittelt. 
Der daraus entstehende Funktions- und Anforderungsumfang dient als Grundlage für das Kapitel \ref{PythonApp}.

\section{Lagerverwaltung 3.0}
Dieses Programm hat ein Anwendungsfenster welches dem Benutzer einen umfassenden Überblick über die LZ gibt.

Das Fenster hat eine Mindest- und Maximalgröße, ist dazwischen aber beliebig skalierbar. 

Dialoge werden genutzt, wenn Einträge gelöscht werden, Verbindungen nicht aufgebaut werden könenn und der Simulationsbetrieb gestartet werden kann.

Felder sind standardmäßig deaktiviert und müssen über Checkboxen aktiviert werden. 

Dateneingaben erfolgen über Dropdown Menüs wenn eine Auswahl festgelegter Werte möglich ist. 
Ansonsten werden Eingabefelder benutzt.

\subsection{Bedienfunktionen}

Der Benutzer kann über das GUI folgende Eingriffe vornehmen:

\begin{itemize}
    \item Die Verbindung zu einem Modbus Server konfigurieren, starten und beenden.
    \item Die Verbindung zur ABB IRC5 Steuerung konfigurieren, speichern, starten und beenden.
    \item In der Produktliste kann gescrollt werden.
    \item In der Inventarliste kann gescrollt werden und es können leere Produkte ein- und ausgeblendet werden.
    \item Die vergangenen Einträge im Eventlogger können gelöscht werden. Im Eventlogger kann gescrollt werden.
    \item In der Lagervisualisierung können an den einzelnen Lagerorten
    \begin{itemize}
        \item Paletten hinzugefügt oder entfernt werden
        \item Becher hinzugefügt oder entfernt werden
        \item Produkt ID und Cup ID der Becher können angezeigt und konfiguriert werden
        \item Durch Klick auf einen Becher werden alle Becher gleichen Produkts in der Lagervisualisierung und auch im
        Inventar blau hervorgehoben.
    \end{itemize}
    \item Palette, Becher und Produkt auf dem Kommissioniertisch können konfiguriert werden
    \item Becher und Produkt auf dem mobilen Roboter können konfiguriert werden.
    \item Der Automatikbetrieb kann gestartet werden
    \item Wenn die Verbindung zum Modbus Server nicht hergestellt wurde kann ein Simulationsbetrieb gestartet werden
    \item Im Simulationsbetrieb kann zudem
    \begin{itemize}
        \item Die Anwesenheit eines mobilen Roboters Simuliert werden
        \item ausgewählt werden, dass der Becher auf dem mobilen Roboter nicht transparent ist (Funktion unklar)
    \end{itemize}
\end{itemize}

\subsection{Informationsdarstellung}

Der Benutzer wird über folgende Inhalte informiert:

\begin{itemize}
    \item Die Verbindungseinstellungen des Modbus Servers und den Verbindungsstatus
    \item Die Verbindungseinstellungen des ABB-Controllers und ihren Verbindungsstatus
    \item Alle möglichen Produkt ID's und die zugehörigen Produktnamen
    \item Produkte und ihre Lagermengen im Inventarbereich.
    \item In der Lagervisualisierung:
    \begin{itemize}
        \item Symbolisierung einer Palette durch dunkelgraues Rechteck, wenn eine Palette vorhanden ist
        \item Symbolisierung der Becher durch weißes Rechteck, wenn ein Becher vorhanden ist
        \item Produknamen und auf Wunsch auch Produkt ID und Becher ID
    \end{itemize}
    \item Im Eventlogger werden folgende Informationen bereit gestellt:
    \begin{itemize}
        \item Fehler und Events der Verbindung über Modbus TCP/IP
        \item Fehler und Events der Verbindung zur Steuerung IRC5 des Industrieroboters
        \item Programmfortschritte und Events im Automatikbetrieb oder der Simulation
    \end{itemize}
\end{itemize}

\section{Controller}

Der Benutzer kann folgende Bedienvorgänge durchführen:

\begin{itemize}
    \item Die Verbindungseinstellungen des Modbus Servers konfigurieren und starten
    \item Transportbefehle können erzeugt werden:
    \begin{itemize}
        \item Transportbefehle eines einzelnen Bechers zwischen Lager, Kommissioniertisch und Roboter
        \item Transportbefehle für einer Palette zwischen Lager und Kommissioniertisch
    \end{itemize}
    Die beiden Befehle können einerseits direkt ausgeführt werden, andererseits können sie auch als Eingabe für die
    Lagerverwaltungssoftware benutzt werden.
    \item Ein Becher in der Andockstation kann mittels einem RFID- Lesegerät beschrieben werden.
\end{itemize}

Der Benutzer wird in diesem Programm nur über seine Eingaben Informiert.
Die Eingaben erfolgen wo möglich über ein Dropdown Menü

\section{RFID Server}

Der Benutzer kann folgende Bedienvorgänge durchführen:

\begin{itemize}
    \item Einen Neuen Listeneintrag erzeugen
    \item Einen oder alle Listeneintrag markieren
    \item Markierte Listeneinträge abwählen
    \item ausgewählte Listeneinträge Löschen, Verbindung starten oder stoppen
    \item Je Listeneintrag kann der Benutzer folgende Einstellungen vornehmen:
    \begin{itemize}
        \item IP und Port des Tag Readers
        \item IP, Port und Modbus Adresse des Endpoints
    \end{itemize}
    \item Ein einzelner Listeneintrag kann gestartet werden
    \item Mit der Taste \glqq Modify \grqq kann ein Listeneintrag zum Bearbeiten freigeschaltet werden
    \item Mit der Taste \glqq Save \grqq können Änderungen gespeichert werden.
\end{itemize}

Der Benutzer sieht in diesem Programme eine Liste aller eingetragenen RFID Reader, Endpoints und Modbus Adressen.
Jeder Listeneintrag kann auf eine Zeile minimiert werden die lediglich den Namen des Endpunkts und die gelesenen
Tag Daten anzeigt.
Im ausgeklappten Zustand sind die Konfigurationen sichtbar, aber grau hinterlegt um anzudeuten, dass die Bearbeitung
gesperrt ist.


