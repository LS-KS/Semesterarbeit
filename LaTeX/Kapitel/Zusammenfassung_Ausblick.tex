% !TeX spellcheck = de_DE
\chapter{Zusammenfassung und Ausblick}

In dieser Arbeit wurde ein breiter Themenbogen gespannt, der von der Analyse der Architektur und Funktion einer C$\#$ / .NET Softwareanwendung bis zur
Konzeptionierung einer neuen Anwendung in Python mit Qt reicht.
In der Auseinandersetzung der Programmiersprachen C$\#$ und Python wurden ihre Unterschiede diskutiert und aufgezeigt, dass eine Portierung 1:1 nicht möglich ist.
Eher handelt es sich dabei um eine Neuentwicklung, die die Funktionen der alten Software übernimmt und um den neuen Kommunikationsstandard OPC UA erweitert.

Mit Augenmerk auf das GUI, wurden wesentliche Informationen und ihre Visualisierung diskutiert und Entwürfe für die neue Anwendung vorgestellt. 
Eine Analyse hat gezeigt, dass der Hauptbildschirm aufgeräumt werden kann, sodass der Hauptbildschirms den Lagerprozess und die Lager-Visualisierung
fokussiert. 
Diejenigen Informationen, die dabei nicht mehr angezeigt werden, können dabei in einem Dialog oder einem neuen Fenster angezeigt werden.
Der Dienst zur manuellen Lagerbedienung wird, genau wie der RFID-Server, als Plugin ausgeführt.

Aspekte der Implementierung wurden diskutiert und veranschaulicht, wie der Aufbau der neuen Anwendung unter Berücksichtigung des MVVC-Konzepts gelingen kann. 
Es wurde detailliert auf die Datenmodellierung eingegangen und die Vorteile der Referenziellen Integrität diskutiert und an Beispielen gezeigt.

Für die Zukunft wurde festgelegt, dass die Weiterentwicklung und Wartung der Anwendung bereits bei der Entwicklung der neuen Anwendung berücksichtigt werden soll. 
Zu dem Zweck werden die einzelnen Controller- und Serviceklassen auf eine enge Kernfunktion beschränkt. 
Die Verwendung der in Python eingebetteten Doc-Strings soll verwendet werden, um mit dem Paket Sphinx eine automatisierte Dokumentation zu erstellen.
Es wurden zudem zwei andere Pakete vorgestellt und diskutiert, mit denen Python-Module getestet werden können, um schnell die Konsistenz des Datenmodells zu überprüfen.
Diese beiden Aspekte sollen es anderen Personen erheblich erleichtern sich in die neue Anwendung einzuarbeiten, um sie zu warten oder zu erweitern.

Der Schwerpunkt des Ausblicks liegt auf der Ideensammlung für kameragestützte Validierungsprozesse in der Lagerzelle.
Es wurden zwei Usecases vorgestellt und Aspekte des Ablaufs diskutiert.
Aus den Ideen wurden Anforderungen an die Kameras und die Implementierung der Software abgeleitet.

%%% Local Variables:
%%% mode: latex
%%% TeX-master: "../MRT-Bericht2020"
%%% End:
